\begin{refsection}
  
  \subsubsection*{Warum sollte man die Stärken und Schwächen seines eigenen Unternehmens und der Konkurrenz kennen?}
Durch realistische Einstufung seines Unternehmens anhand einer SWOT-Analyse kann man einfacher strategisch günstige Entscheidungen treffen, um auf dem Markt erfolgreicher zu werden.
Man führt die Stärken und Schwächen eines Unternehmens auf und kann dadurch Chancen besser nutzen und Risiken minimieren. Eine SWOT-Analyse hilft dabei, zukunftstragende Entscheidungen zu treffen, die Konkurrenten direkt miteinbeziehen.
Dabei analysiert man das eigene Unternehmen und die wichtigsten Konkurrenten, als auch aufstrebende Gegenspieler auf interne und externe Faktoren.
Nachteile bringt eine korrekt durchgeführte Konkurrenzanalyse keine, sondern bietet einen konkreten Ist-Zustand, welchen man in wichtige wirtschaftliche Entscheidungen miteinbeziehen sollte.
Eine falsch durchgeführte Analyse kann jedoch schlechte Entscheidungen zur Folge haben, also sollte man diese bedächtig und ehrlich ausarbeiten.

  \clearpage
  \printbibliography[heading=subsubbibliography]
\end{refsection}
\clearpage
