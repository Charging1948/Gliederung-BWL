\begin{refsection}
  
  \enquote{\textit{[A]m Ende ist es immer der Kunde, der durch seine Kaufentscheidung über den Erfolg oder Misserfolg eines Unternehmens am Markt entscheidet}} \autocite[76]{herp1990anders}
  
  Noch vor einigen Jahrzehnten war eine Segmentierung des Kundenstamms ungebräuchlich. Es war üblich standardisierte Produkte, die grö\ss{}tmögliche Kundengruppen ansprechen, auf den Markt zu bringen. Doch im Laufe der Zeit forderte die zunehmende Individualisierung der Menschen immer mehr spezielle und individualisierte Produkte und Dienstleistungen \autocite[1]{Rudolph2009}.

Da es einem Unternehmen nicht möglich ist jeden Kunden einzeln anzusprechen, gilt es bei der Kundensegmentierung bestimmte Gruppen an Kunden besonders anzusprechen und für das eigene Produkt oder Dienstleistung zu  überzeugen. Dies geschieht z. B. durch spezielle Angebote oder gezielte Marketingstrategien. Ein weites, wichtiges Ergebnis ist auch die langfristige Bindung der der Kunden an das Unternehmen und dessen Produkts \autocite[1]{Rudolph2009}. Um solche Gruppen an Kunden erstellen zu können werden die Methoden der Kundensegmentierung verwendet.

Hierbei gilt es die Kunden nach bestimmten Merkmalen in Segmente zu unterteilen. Diese Merkmale können unter anderem soziodemografisch, geografisch, psychografisch und verhaltensorientiert sein. Daraufhin werden anhand einem, oder mehreren Merkmalen die Kundensegmente gebildet und anhand dieser Gruppen Methoden entwickelt, jeweils diese Gruppen anzusprechen. 

Ein gutes Beispiel für eine gelungene Kundensegmentierung wäre die Hotelkette Hyatt. Durch eine Analyse des Kundestammes konnte herausgefunden werden, dass ein gro\ss{}er Teil der Kunden das Hotel kostenorientiert aussuchen. Um dieses Segment an Kunden anzusprechen und für längere Zeit an die Dienstleistung des Unternehmens zu binden können diesen Kunden ein Sonderangebot gemacht werden, welches ihnen Rabatt auf den nächsten Aufenthalt in einem Hotel der Kette Hyatt gewährt. Dadurch können die Kosten des nächsten Aufenthalts im Vergleich zu anderen Anbietern, deutlich sinken, was den kostenorientierten Kunden dazu ermutigt, das nächste Mal wieder ein Hotel von {Hyatt} zu buchen \autocite[4-7]{dissertationprime2015}.
  \clearpage
  \printbibliography[heading=subsubbibliography]
\end{refsection}
\clearpage
