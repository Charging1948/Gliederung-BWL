
\begin{refsection}
Der Begriff Lead Management wird in der Literatur unterschiedlich definiert. Dies ist nicht verwunderlich. Betrachtet man die Einordnung des Begriffs, wird man entweder im Marketing oder im Vertrieb fündig. Die Sichtweisen variieren somit je nach Fokus zwischen einer marketingzentrierten und einer vertriebszentrierten Definition. \autocite[301]{Wenger2021}

Lead Management im Allgemeinen ist ein kritischer Prozess im Vertrieb, bei dem aus potenziellen Interessenten zufriedene Kunden werden. Lead Management umfasst Schritte wie die Identifizierung und Generierung von Leads, die Qualifizierung von Leads, die Kontaktaufnahme und Bedarfsanalyse, die Vorbereitung des Verkaufsabschlusses, den Verkaufsabschluss und die Übergabe an den Kundenservice sowie die Kundenbindung und Nachbetreuung. Eine effektive Leadbearbeitung kann dazu beitragen, die Verkaufschancen zu maximieren und den Unternehmenserfolg zu steigern.

Leadmanagement ist ein zentrales Thema. Es ist sowohl für kleine Start-Ups als auch für gro\ss{}e Unternehmen von gro\ss{}er Bedeutung für den Erfolg. Dabei ist besonders auf eine gute Zusammenarbeit zwischen Vertrieb und Marketing zu achten.

Lead Management beschreibt den Prozess, in dem potenzielle Kunden identifiziert, qualifiziert und zu zufriedenen Kunden gemacht werden. Diesem Prozess kommt eine hohe Bedeutung zu, da er wesentlich zur Umsatzgenerierung und Kundenbindung beiträgt.

Ein gut durchdachter Lead Management Prozess kann dazu beitragen, die Konversionsrate zu erhöhen und somit mehr Umsatz zu generieren. Darüber hinaus kann durch gezieltes Lead Nurturing die Kundenzufriedenheit gesteigert werden, da die Kunden das Gefühl haben, dass ihre Bedürfnisse verstanden und berücksichtigt werden.
  \clearpage
  \printbibliography[heading=subsubbibliography]
\end{refsection}
\clearpage
