\begin{refsection}
  
  
  Durch das Internet hat sich vieles verändert. Man chattet mit Freunden, findet Dates und kauft online ein. Warum sollte man also noch vor Ort einkaufen, wenn alles einen Onlineshop hat?

  Natürlich gibt es noch viel mehr Gründe, warum ein guter Standort wichtig ist. Lieferketten finden immer noch in der Realität und nicht online statt und Fachkräfte, obwohl das Angebot für Home-Office steigt, wählen oft Jobs in ihrer Nähe. Steuern und standortbedingte Vorteile können durch einen Onlineshop nicht ersetzt werden. Andere Kriterien sind nicht so eindeutig, wie zum Beispiel Werbung und Sichtbarkeit. Muss man dieses Kriterium bei der Standortwahl noch beachten, oder übernimmt Online-Werbung diese Aufgabe?

  Die Standortwahl hängt von sehr vielen Faktoren ab, die für verschiedene Unternehmensmodelle mehr oder weniger relevant sein können, was es umso interessanter macht, herauszufinden, wie wichtig der Standort eigentlich noch ist.
  \clearpage
  \printbibliography[heading=subsubbibliography]
\end{refsection}
\clearpage
