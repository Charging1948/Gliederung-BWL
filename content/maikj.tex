\begin{refsection}
  
  
  Stellen Sie sich folgende Situation vor:

  Sie sind Marketingleiter eines mittelständigen Unternehmens und wollen ihr neues Produkt, eine Brille mit eingebautem Hörgerät, vermarkten. Ihre firmeninterne Forschungsabteilung hat an diesem Produkt nun die letzten zehn Jahre geforscht und sind sich sicher, dass diese Brille mit eingebautem Hörgerät die gesamte Branche revolutionieren wird. Als Hauptwerbeplattform wählen Sie die Social Media Plattform TikTok, da man dort sehr schnell viel Aufmerksamkeit erhalten kann. Es wird also eine teure Werbekampagne gestartet, doch diese bleibt größtenteils erfolglos. Warum fragen Sie sich jetzt? Es stimmt zwar, dass man über TikTok, besonders über Kooperationen mit Influencern, schnell viel Aufmerksamkeit bekommt, allerdings ist die Nutzungsquote der App bei jüngeren Generationen viel höher als bei Älteren \autocite{Statista1, Statista2}. Eine Brille mit eingebautem Hörgerät ist nichts, was diese Generationen brauchen und damit auch kaufen würden. Sie haben also viel Geld für eine äußerst ineffiziente Werbekampagne ausgegeben und ihr Produkt hat sich kaum verkauft.

  Dieser Ausgang hätte sich ganz einfach mit einer guten Kundenanalyse vermeiden lassen. Hätten sie vorher eine ausführliche Kundenanalyse durchgeführt, dann hätten sie gewusst, dass sich ihre Kundengruppe größtenteils aus älteren Personen zusammensetzt und ein anderes, von dieser Altersgruppe mehr benutztes, Werbemedium wählen können.

  Als weiteres Beispiel für Kundenanpassung musste sich die Rittal GmbH \& Co. KG schnell den Anforderungen der US-Kunden anpassen. So erkannte der damalige US-Chef Gregg Holst, dass die US-Kunden weniger Wert auf die entstehenden Kosten über die gesamte Produktlebensdauer legten, sondern auf den Einkaufspreis des Produktes. Ebenso war die sofortige Lieferung ein wichtiger Punkt für die US-Kunden. Als Antwort darauf musste Rittal die bisher erfolgreiche Vertriebsstrategie anpassen, was ihnen am Ende auch gelungen ist \autocite[66]{Scheed2021}. 

  Auch in diesem Beispiel wird die Relevanz einer guten Kundenanalyse klar. Ebenfalls kann man sehen, dass die Kundenanalyse nicht nur im B2C-Markt, sondern auch im B2B-Markt eine wichtige Rolle spielt. Heutzutage reicht die reine Fokussierung auf hochwertige Produkte im globalen Markt nicht mehr aus. Stattdessen werden die Kundenbindung und möglichst große Individualisierung auf die Kunden, auch im B2B-Markt, immer wichtiger \autocite[66]{Scheed2021}. 

  \clearpage
  \printbibliography[heading=subsubbibliography]
\end{refsection}
\clearpage
