\begin{refsection}
  
  Warum sollte der Standort wichtig sein, wenn ich ein Unternehmen gründe? Sollte nicht eher das Konzept oder die Idee, bzw. das Produkt im Vordergrund stehen? 

  Wenn man an die Gründung eines Unternehmens denkt, schenkt man der Standortauswahl vielleicht eine nicht sehr große Bedeutung. Aber die Wahl des Standortes sollte nicht unterschätzt werden, denn der Standort hat großen Einfluss auf Produktionskosten, den Absatzmarkt oder das Image des Unternehmens. Man kann also mit einer sorgfältigen Wahl des Standortes viele Vorteile auch der Konkurrenz gegenüber erhalten und dadurch die eigene Wettbewerbsfähigkeit steigern.

  Es gibt einige Standortfaktoren, die einem helfen einen passenden Standort für das eigene Unternehmen zu finden, da die Standortentscheidung sehr individuell und abhängig von der Auslegung des Unternehmens ist. Für ein Unternehmen, das z.B. Stahl produziert, ist eine Nähe zu Ressourcen wie Kohle wichtiger, da lange Transportwege verhindert werden. Das Unternehmen erhält zudem einen Vorteil gegenüber der Konkurrenz, die weiter entfernt von den Ressourcen angesiedelt sind, denn durch die höheren Transportwege entstehen für sie mehr Ausgabekosten. Hingegen spart das Unternehmen nahe den Ressourcen an Transportkosten und kann das Produkt günstiger anbieten. 

  Für  Firmen anderer Branchen dagegen spielen andere Standortfaktoren eine größere Rolle. Ein IT-Unternehmen, das z.B. auf Software-Entwicklung spezialisiert ist, benötigt, anders als in der Stahlindustrie, keine Ressourcen wie Kohle. Allerdings sind qualifizierte Arbeiter für ein solches Unternehmen wichtig und eine moderne Infrastruktur. Daher bietet sich beispielsweise ein Standort in der Nähe moderner Städte mit guter Infrastruktur, flächendeckendem Internet oder Universitäten für junges, gut ausgebildetes Personal an. 

  Eine gute Standortauswahl sollte nicht in ihrer Bedeutung unterschätzt werden und sorgfältig abgewogen werden, um den größtmöglichen Nutzen zu erhalten und die eigene Wettbewerbsfähigkeit zu steigern.
  \clearpage
  \printbibliography[heading=subsubbibliography]
\end{refsection}
\clearpage
