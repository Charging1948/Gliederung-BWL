\begin{refsection}
  
Verschiedene Werbemittel werden auf unterschiedliche Arten eingesetzt, um Zielgruppen besonders wirksam zu erreichen. Neben den verschiedenen Medien, die von Werbemitteln abgedeckt werden können, gilt es auch, die Anwendung verschiedener Werbemittel innerhalb desselben Mediums abzustimmen. Daher handelt dieser Abschnitt von den verschiedenen Anwendungsmöglichkeiten der verschiedenen Werbemittel. Um eine erfolgreiche Werbekampagne zu planen, ist es wichtig, die verschiedenen Anwendungsmöglichkeiten der Werbemittel zu verstehen. 

Beispielsweise kann eine Zielgruppe durch eine Anzeige in einer Zeitung oder Zeitschrift erreicht werden, während eine andere Zielgruppe eher  auf Online-Anzeigen oder Social-Media-Werbung anspricht.

Insgesamt ist es entscheidend, dass sorgfältig abgewogen wird, welche Werbemittel am besten geeignet sind, um ihre Zielgruppen zu erreichen. Eine effektive Werbekampagne erfordert sorgfältige Planung und Abstimmung verschiedener Werbemittel und Medien, um sicherzustellen, dass die gewünschten Ergebnisse erzielt werden.

Au\ss{}erdem sollte beachtet werden, dass eine sinnvolle Wahl von Werbemittel und Medien auch von der Art des zu bewerbenden Produkts oder der Dienstleistung abhängt. Eine gezielte und zielgruppengerechte Auswahl der Werbemittel und Medien kann dazu beitragen, dass die Werbekampagne ihre Ziele erreicht und eine höhere Effektivität erzielt.

Es wird konkret darauf eingegangen, welche Zielgruppen von den verschiedenen Werbemitteln besonders effektiv erreicht werden können, wie effektiv verschiedene Medien auf verschiedene Zielgruppen sind und welche Werbemittel - innerhalb eines gemeinsamen Mediums - welche Effekte auf die Zielgruppen haben.

  \clearpage
  \printbibliography[heading=subsubbibliography]
\end{refsection}
\clearpage
