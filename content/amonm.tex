\begin{refsection}
  
  
  In der heutigen Welt ist Werbung nicht mehr wegzudenken. An jeder Hausecke steht eine Litfasssäule, eine Neonreklame oder nur ein and die Wand geklebtes Plakat. In jedem Haus laufen Werbespots über die Fernsehbildschirme, Banner blockieren die halbe Internetwebsite, bunte Reklamezeitschriften und Kataloge kommen jeden Tag an die Tur geliefert und auch wenn es grö\ss{}tenteils ein Relikt vergangener Zeiten ist kommt doch auch noch der eine oder andere Vertreter im Au\ss{}endienst zum klingeln vorbei.\\
  In dieser Werbebranche sind eine Menge Begriffe unterwegs, manche professioneller Natur wie Kundenanalyse, AIDA Prinzip oder Branding. Aber es gibt auch nachgesagte Begriffe wie Lügenpresse, Geldmacherei und Manipulation. Aber sind diese nachgesagten Begriffe wirklich so prevalent wie es einige Leute einem glauben machen wollen?\\
  Eine gute Werbung sollte den Kunden allem voran Informieren. Kann er das Produkt auch wirklich verwenden?\\

  In der Realität sieht das aber etwas anders aus. Die Werbebranche ist eine gigantische Industrie einzig und allein darauf aus den Kunden zum Kauf anzuregen. Aber wie weit kann dieses Anregen gehen bis die Grenze errreicht ist? Wo genau liegt diese Grenze? Wie ist sie definiert? Wie hat sich die Werbung im Verlauf der Zeit geändert? Und werden die heutigen Kunden in den vielen bunten Werbespots belogen?\\
  Sogar dieser Text in der Gleiderung selbst ist eine Werbung. Das Ziel ist in diesem Fall die Aufmerksamkeit auf den Text zu richten und die Leser dazu zu bringen diesen Text zu lesen. Allerdings sind hier die Rahmenbedingung anders als in z.B. der Telewerbung.\\
  Wo hier Wert auf den Inhalt und das Lesen eben jenes gelegt wird, da dieser das Ziel des Textes ist, wird in der Werbung ein höherer Wert auf den Kauf und die Anregung zu der Entscheidung dazu gelegt. Would you like to know more?
  \clearpage
  \printbibliography[heading=subsubbibliography]
\end{refsection}
\clearpage
