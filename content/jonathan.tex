\begin{refsection}
  
  \subsubsection*{Wer sind Vertriebspartner?}
  Vertriebspartner sind externe Unternehmen, die im Namen eines Anbieters arbeiten und ihm bei seinen Vertriebsaufgaben unterstützen. Folglich sind es Händler oder Vermittler, die im Auftrag eines Dritten Waren oder Dienstleistungen verkaufen.
  \subsubsection*{Warum ist das Thema relevant/spannend?}
  Vertriebspartner sind ein wichtiger Faktor für den Erfolg von Unternehmen, insbesondere im Verkauf von Produkten und Dienstleistungen. Sie können ein Unternehmen helfen, sich schnell auf dem Markt auszubreiten und deshalb ist es wichtig, die richtigen und passenden Vertriebspartner auszuwählen. Ein wesentlicher Bestandteil des Vertriebs ist es Kunden zu erwerben und deren Beziehungen zu Pflegen. Vertriebspartner spielen dabei eine wichtige Rolle, da sie dazu beitragen.

  Es gibt verschiedene Arten von Vertriebspartnern, z.B. Gro\ss{}händler, Handelsvertreter, Franchise-Partner, usw. Jede Art von Vertriebspartner hat ihre eigenen Vor- und Nachteile.

  Die Auswahl der besten Vertriebspartner erfordert eine Analyse und Bewertung verschiedener Kriterien, wie z.B. Kosten, Reichweite, Zielgruppen, Kundenerfahrung und Marktposition.

  \subsubsection*{Praxisbeispiele}
  Apple: verkauft seine Produkte hauptsächlich über Einzelhändler wie Best Buy und über seine eigenen Apple Stores. Es nutzt auch Online-Marktplätze wie Amazon und eBay. 

  Amazon: hat selbst ein Online Markt-Marktplatz, der es anderen Unternehmen ermöglicht ihre Produkte zu verkaufen und viele Käufer zu erreichen.
  \clearpage
  \printbibliography[heading=subsubbibliography]
\end{refsection}
\clearpage
