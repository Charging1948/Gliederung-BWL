\begin{refsection}
  
  Das Thema „Arten der Konkurrenzanalyse“ ist relevant, um von den vielen verschiedenen Wegen Informationen über die Konkurrenz herauszufinden, die die meisten konkurrenz-wichtigen Aspekte der Gegner-Firma ermitteln. In diesem Unterpunkt der Gliederung werden daher die verschiedenen Arten wie zum Beispiel die Analyse der Produkte oder der Lieferanten und dort die vorhandenen Schwächen zu erkennen und sie in seiner eigenen Firma zu vermeiden oder sie sogar zu nutzen um der Gegner-Firma zu schaden. Ich persönlich habe mich für dieses Thema entschieden, da es ein wichtiger Punkt in dem Unternehmen darstellt, um unter anderem entscheiden zu können, ob man am Wettbewerbsmarkt teilnehmen kann oder ob man dort scheitert, weil zum Beispiel die Schwächen der Konkurrenz zu gering sind und sie dadurch ein Monopol aufgebaut haben. Wenn man die Schwächen und Stärken der Konkurrenz jedoch weis ist es einfacher seine Vorgehensweise in der Zukunft im Wettbewerb planen zu können, um der scheinend monopol-stehenden Gegner-Firma mithalten zu können oder sie sogar zu überholen. Um die Analyse durchzuführen ist es jedoch wichtig verschieden an diese heranzugehen und es aus mehreren Winkeln zu betrachten. Zum einen kann man Kundenbefragungen durchführen aber auch was eher wichtig ist, ist die online Recherche die man betreiben kann, um mehr über die Firma zu erfahren. Im Folgenden sollen nun diese Methoden unter anderem erklärt werden und auch Alternativen dargestellt werden.

  \clearpage
  \printbibliography[heading=subsubbibliography]
\end{refsection}
\clearpage
