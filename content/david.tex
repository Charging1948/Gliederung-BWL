\begin{refsection}
  
  Wenn eine Idee zu einer unternehmerischen Handlung wird und man sich erste Gedanken um das weitere Vorgehen macht, kommt man nicht umher, sich intensiver mit dem Markt und dem Wettbewerb zu beschäftigen. Dabei sollte man neben vielen anderen Faktoren die Aufmerksamkeit auf Kunden und Konkurrenz legen. Schlie\ss{}lich sind es eben jene, die im weiteren Verlauf den Erfolg eines Unternehmens ausmachen. Eine Konkurrenzanalyse ist ein sehr wichtiger Teil davon und beschäftigt sich damit, ihre Position im Markt zu bestimmen und Wettbewerbsvorteile zu stärken. Und genau deswegen ist es wichtig, sich so früh wie möglich die Frage zu stellen, wer die Konkurrenten sind und wie man diese am besten herausfiltert. Denkt man zurück an die Geschäftsidee muss man nun damit anfangen, den Markt zu analysieren und herausfinden, ob das eigene Produkt bzw. die Dienstleistung überhaupt benötigt wird und einen Mehrwert bei den Kunden liefert. Zeitgleich überprüft man welche Unternehmen ähnliche Produkte/Dienstleistungen anbieten und wo deren Stärken und Schwächen liegen. Nur so kann man Strategien erarbeiten, um am Markt bestehen zu können.

  Hat das ganze nun Priorität für mich als Unternehmer? Diese Frage lässt sich leicht mit einer Analogie mit Informatikbezug beantworten: Man stelle sich vor, ein neues proprietäres Betriebssystem entwickeln und erfolgreich vertreiben zu wollen. Dabei soll das Betriebssystem viele allgemeine Funktionen anbieten. Wieso klingt diese Idee so formuliert erstmal schwierig umzusetzen? Die meisten würden hier antworten: Es gibt bereits zwei sehr dominante Betriebssysteme auf dem Markt, die von Microsoft und Apple hergestellt werden. Diese Firmen gehören zu den erfolgreichsten der Welt und Sie entwickeln die Betriebssysteme bereits mehrere Jahrzehnte. Was hier für einen Informatiker intuitiv klingt, ist ein sehr einfaches Beispiel für eine Konkurrenzanalyse, die relevante Konkurrenten hervorgebracht hat. Wird das Produkt nun spezifischer als ein Betriebssystem oder gehört in andere Branchen, muss die Konkurrenzanalyse viel detaillierter durchgeführt werden. Dabei entscheidet schon die richtige Herausarbeitung der relevanten Konkurrenten über einen Misserfolg oder Erfolg des Unternehmens!


  \clearpage
  \printbibliography[heading=subsubbibliography]
\end{refsection}
\clearpage
