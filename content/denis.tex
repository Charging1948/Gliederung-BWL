
\begin{refsection}
  
Grundsätzlich beschreibt die Preisstrategie, Überlegungen und Entscheidungen, die den Verkaufspreis eines Produkts oder einer Dienstleistung betreffen und gehört somit zu der Preispolitik.

Ausgangspunkt ist bei vielen Preisstrategien die Preisuntergrenze, welcher nur die anfallenden Kosten deckt. Des Weiteren bestimmen sowohl externe- als auch interne Faktoren, wie die Unternehmensziele und Positionierung eine Entscheidung bei der Preisgestaltung. Je nachdem welches Image durch den Preis des Produkts vermittelt werden soll.

Die verschiedenen Arten der Preisstrategien machen das Thema sehr abwechslungsreich, interessant und spannend. Hier gibt es zum Beispiel die Festpreisstrategie, welches sich auf einen fixen Preis eines Artikels beschränkt. Dieser wird einmal für das jeweilige Produkt festgelegt und wird nicht geändert. Durch diese Strategie spielt die Positionierung Ihres Produkts auf dem Markt eine gro\ss{}e Rolle.

Des Weiteren gibt es eine Niedrigpreisstrategie, bei der man sich mit besonders günstiger Preis Positioniert. Als Gegenstück davon, könnte durch die Verwendung der Hochpreisstrategie ein besonders hoher Preis verlangt welcher der Qualität und dem Image der Firma entsprechen sollte. Bei der Luxuspreisstrategie würde einem als erstes Gucci, Rolex oder Louis Vuitton in den Sinn kommen, bei der es mit Exklusivität und einem kleinem Kundenpreis ein starkes Prestige verdeutlicht.

Das Ziel jeder Preisstrategie kann stark von den Ambitionen und Ziele der Firma variieren, jedoch haben die meisten ein Ziel, welcher der Profit wäre. Der Gro\ss{}teil der Firmen möchte einen möglichst gro\ss{}en Gewinn erzielen, welcher durch die passende Preisstrategie maximiert werden kann. Jedoch ist nicht gesagt, dass jede Preisstrategie zu jeder Firma passt. 

Da ich selbst verschiedene Preisstrategien im alltäglichen Leben schon probiert und untersucht habe, fand ich das Thema sehr passend zu mir. Hauptsächlich möchte ich meinen Wissenstand mit dieser Gelegenheit erweitern und es nutzen um mehr Gewinn zu erzielen.
  \clearpage
  \printbibliography[heading=subsubbibliography]
\end{refsection}
\clearpage
