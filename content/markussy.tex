\begin{refsection}
  
  \subsubsection*{Warum dem Online-Marketing in Zukunft einen so wichtigen Stellenwert zuweisen?}
  Wenn Firmen heutzutage an ihre Kunden herantreten wollen, passiert das nicht mehr per Brief oder ähnlichem, sondern meist per Mail oder Social-Media.

Die Frage nach dem \enquote{Warum?} lässt sich hierbei ganz leicht beantworten und auf wenige Punkte rationalisieren, es ist günstiger, personalisierter und man erreicht mehr Menschen über das Internet.

Für den Kunden bieten sich aber auch Vorteile, so ist es möglich eine grö\ss{}ere Bindung zum Unternehmen aufzubauen, da einem Personalisierte, relevante und informative Inhalte individuell bereitgestellt werden können und dadurch eine Vertrauensbasis geschaffen
werden kann.

Ein anderer Punkt der für die Relevanz spricht ist die Messbarkeit und Reaktion auf laufende Kampagnen, so kann man Live verfolgen wie der Markt auf spezielle Werbema\ss{}nahmen reagiert und kann bei bedarf schnell eingreifen um bessere Ergebnisse zu erzielen.
Gerade durch Corona haben wir alle gemerkt wie schnelllebig unsere Welt ist, bis die Plakate für eine Sicherheitsma\ss{}nahme gedruckt worden sind kann es sein dass durch eine neue Variante diese wieder au\ss{}er Kraft gesetzt wurde.

Damit so etwas einem Unternehmen nicht mehr passiert, hier eher mit Produkten oder Dienstleistungen, brauchen die Firmen verstärkt den Einsatz von Online-Marketing und die die noch keine Strategie entwickelt haben sollten dies schleunigst tun, damit sie der Rest nicht abgehängt.
  \clearpage
  \printbibliography[heading=subsubbibliography]
\end{refsection}
\clearpage
