\begin{refsection}
  
  Der Begriff der Preisbildung lässt sich als Prozess definieren, bei dem sich ein Unternehmen auf einen Preis für eine Dienstleistung oder ein Produkt festlegt. Das Ziel des Prozesses ist einen Preis zu finden, der sowohl für das Unternehmen als auch für seine Kunden akzeptabel ist und den Gewinn des Unternehmens steigert. 

  Wichtig zu beachten ist der Marktpreis der für das angebotene Gut vorliegt. Der Marktpreis ist der durchschnittliche Preis zu dem ein bestimmtes Gut zu einem bestimmten Zeitpunkt am Markt gehandelt wird. Weicht man mit dem Preis für sein Gut zu sehr vom Marktpreis ab, kann dies negative Folgen auf den Gewinn haben den man mit dem Gut erzielt.

  Setzt man beispielhaft den Preis zu hoch an, werden Kunden c.p. eher bei den Konkurrenten einkaufen, da diese das Gut für weniger Geld anbieten. Setzt man ihn zu niedrig an, mindert man den eigenen Gewinn an dem Verkauf von dem Gut.

  Ebenfalls zu beachten ist die Marktform die für das Gut existiert. Es gibt auf jeder Marktseite (Angebot und Nachfrage) drei Konstellationen die auftreten können: Monopol, Oligopol und Polypol. Das Polypol beschreibt die Existenz von vielen konkurrierenden Marktteilnehmern, das Oligopol beschreibt die Existenz von einigen wenigen konkurrierenden Marktteilnehmern und beim Monopol gibt es nur einen einzigen Marktteilnehmer auf einer Marktseite.

  Ist das Gut einzigartig und wird nur vom preisbildenden Unternehmen angeboten, spricht man von unvollständigem Wettbewerb und einem Angebotsmonopol. Das Unternehmen hat in diesem Fall mehr Freiheit was die Festlegung auf einen Verkaufspreis betrifft, da es keine Konkurrenten gibt die Einfluss auf den Marktpreis nehmen.

  Sind hingegen viele Anbieter für ein Produkt am Markt, spricht man von vollständigem Wettbewerb und einem Angebotspolypol. In diesem Fall ist die Freiheit des Unternehmens bei der Preisbildung eingeschränkt.

  Die oben aufgeführten Punkte sind nur ein kleiner Teil der Theorie hinter der Preisbildung, es spielen noch viel mehr Faktoren eine Rolle bei der Festlegung auf einen Preis für ein Produkt.

  Ausgewählt habe ich die Preisbildung weil ich schon länger mit dem Gedanken spiele eine Dienstleistung anzubieten (Erstellung von personalisierten Webseiten). Bisher davon abgehalten hat mich hauptsächlich die Tatsache, dass ich mir unsicher bin zu welchem Preis ich meine Dienstleistung anbieten kann. Daher erschien mir dies als perfekte Gelegenheit um mich mit Preisbildung auseinanderzusetzen.
  % \cite{Forner2022} \cite{MargaretaKulessa2021} \cite {Pollert2016}
  \clearpage
  \printbibliography[heading=subsubbibliography]
\end{refsection}
\clearpage
