\begin{refsection}
  
Jeder hat es schonmal mitbekommen, man bestellt etwas auf Amazon und erhält kurz darauf eine E-Mail, mit der bitte seine Kauferfahrung zu bewerten. Hier wird von Amazon mithilfe einer Online-Befragung eine Kundenanalyse durchgeführt, mit welcher Amazon herausfinden will, wie zum Beispiel das Kauferlebnis noch einfacher und effizienter für den Kunden gestaltet werden kann. 

Die Kundenanalyse ist ein wichtiger Bestand jedes Unternehmens heutzutage. Sie entscheidet darüber, wie die nächste Marketingkampagne aussieht, oder wie die Verpackung eines Produktes aufgebaut ist, denn durch sie werden die Bedürfnisse und auch die Zufriedenheit der Kunden festgestellt. Ohne dieses Feedback wäre es unmöglich ein Produkt erfolgreich weiterzuentwickeln, da man nicht wüsste, was die Kunden wollen und was sie nicht wollen.

Die Methoden, welche verwendet werden, um die Kundenanalyse durchzuführen, spielen hier eine gro\ss{}e Rolle. Sie entscheiden an welcher Zielgruppe die Analyse durchgeführt wird, also ob hier zum Beispiel nur Rentner oder nur Jugendliche angesprochen werden, oder auch ob man nur Bestandskunden oder potenzielle Kunden analysieren möchte.

Ich habe mich für dieses Thema entschieden, da ich es sehr interessant finde, wie eine Kundenanalyse je nach verwendeter Methode komplett unterschiedliche Ergebnisse liefern kann. Es hilft mir zum Beispiel dabei die richtigen Methoden auszuwählen, wenn ich zum Beispiel, Feedback über ein geschriebenes Programm zu erhalten, damit ich diese verbessern kann.
  \clearpage
  \printbibliography[heading=subsubbibliography]
\end{refsection}
\clearpage
